\documentclass{article}
\usepackage{amsmath}

\begin{document}

If we have a 4 atom cell that can be populated with atoms of the
following concentrations $(1,2,1)$ and the following site restrictions:
\begin{equation}
\begin{pmatrix}
  1 & 2 & 3 & 4\\
  0 & 0 & 0 & 0 \\
  1 & 1 & 1 & 1 \\
    &   & 2 & 2 \\
\end{pmatrix}
\end{equation}
where the top row indicates the site and the following rows indicate
the atomic species that can occupy that site. In this case the 3rd
atomic species cannot occupy the first two lattice sites. The code
will then proceed by finding the unique arrangements of the first
atomic species, which has the following possible arrangements:

\begin{equation}
  \begin{matrix}
  (0, & \bullet, & \bullet, & \bullet) \\
  (\bullet, & 0, & \bullet, & \bullet) \\
  (\bullet, & \bullet, & 0, & \bullet) \\
  (\bullet, & \bullet, & \bullet, & 0) \\
  \end{matrix}
\end{equation}

Of these only the first $(0,\bullet,\bullet,\bullet)$ is unique. Then the
second atomic species is placed, (which also fixes the occupasion site
of the third species) resulting in the following possible arrangements:
\begin{equation}
\begin{matrix}
  (0, & 1, & 1, & 2) \\
  (0, & 1, & 2, & 1) \\
  (0, & 2, & 1, & 1) \\
\end{matrix}
\end{equation}

Of these two only the first and the second are symmetrically unique,
the second being equivalent to the first.

The issue becomes that the 3rd arrangement $(0,2,1,1)$ is unique but
violates the site restrictions of the system. Enum3 then included the
arrangement $(1,1,2,0$) since it is unique and equivalent to
$(0,2,1,1)$, however, enum4 will never consider the arrangement
$(1,1,2,0)$ due to the equivalence of any placement of the first atom
on the lattice.

The group for this system is:
\begin{equation}
\begin{matrix}
  (1, & 2, & 3, & 4) \\
  (2, & 1, & 4, & 3) \\
  (1, & 2, & 4, & 3) \\
  (2, & 1, & 3, & 4) \\
  (3, & 4, & 1, & 2) \\
  (4, & 3, & 2, & 1) \\
  (4, & 3, & 1, & 2) \\
  (3, & 4, & 2, & 1) \\
\end{matrix}
\end{equation}

just so you know.


\end{document}
